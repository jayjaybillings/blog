% Copyright 2024- Jay Jay Billings. Some rights reserved.
\documentclass{article}
\usepackage{graphicx} % Required for inserting images
\usepackage{hyperref}
\usepackage{tabularx}

\title{So, COVID sucks}
\author{Jay Jay Billings, Ph.D.}
\date{\today}

\begin{document}

\maketitle
If you noticed that I dropped off the face of the planet last week, it was because I tested positive for SARS-COV-2, the virus that causes COVID-19. I'll call it all COVID, though, since it seems almost no one else distinguishes between the two. 

I've been presumed positive for COVID twice before, but I never had a positive PCR or antigen test. The first time I was presumed positive was at the end of May 2022 when I found myself bedridden and going through something that I have to imagine a bad death would be like. I had no idea what was happening to my body. I felt like my airway was collapsing, my throat burned like desert sand, and I had this really weird, terrible feeling that I vaguely remembered from undergrad. That feeling was cold deep in my bones - like they were blocks of ice themselves! It was a fever of 102.5, and my first recorded fever in nineteen years.

I thought it was time to head to the hospital, which was a great way to completely blow \$350. UT Hospital offered me some steroids and sent me on my way. The overworked third-shift night doctor put her hand on her hip, said my airway was clear, so I'd be fine, and implied that I was being a baby while she handed me my paperwork with her other hand. They wouldn't test me for COVID or flu. My own doctor, who was fantastic, saw me a week later after I could get out of bed to ask him why I lost my voice. Someone in our family had tested positive for flu and COVID the night before I wasted my gas, so he thought it was likely one or the other. While my fever only lasted a short time, which is more a flu behavior (``flu likes to burn through you,'' my doctor said), COVID seemed the better fit because I couldn't smell my kids' poopy diapers, and foods started tasting weird. He did a very uncomfortable PCR test where the swab felt like it was tapping on the back of my head, and then we had a long chat about life, the universe, and everything, including COVID, while we waited on the negative test results. We expected it to be negative because I was clearly on the mend. My voice returned about three and a half weeks later. My sense of smell might be back, but I'm not sure. I don't know if tuna will ever taste the same.

The second time I was presumed positive for COVID was this holiday season during the last week of December 2023 and the first week of January 2024, where, once again, I had the tell-tale symptoms described above: a gnarly fever, smells vanished, food tasted strange, going to the hospital seemed like a great idea, etc. I could barely move, and just getting around the house felt like the world was ending. The best effort I could muster to watch the kids was to put a bean bag chair in front of the playroom door and collapse in a heap that blocked their escape. I met with my new FNP in VA about the experience the following week when I could still only hold myself up for a little, and we thought it was extremely likely that I was COVID-positive. 

Despite churning through tests like butter, I never tested positive for COVID in either of these cases. I started to figure maybe I was one of those people who never tests positive until last week when I finally did test positive on a COVID test! I was in complete shock when the red line came into view. After the shock, I felt embarrassed and angry because I had been out around town thinking I was just sniffing back some allergies when I went to my exam and grabbed a bite during my lunch break. It was four years minus one day since I had my first work trip canceled due to this new SARS-COV-2 virus - A good run without a confirmed, positive test! There was no waffling in presumption anymore: I had COVID.

I was feeling good that day, although with maybe a touch of malaise. I started sneezing in the morning after some light sniffles, and then my nose started running. After blaming it on allergies one more time, I started thinking maybe it wasn't. I figured I'd take a test to see since malaise usually means I'm about to get rocked, and lo! there was my first positive COVID test. I took thirty minutes or so to get my affairs in order (including packing away a mostly written blog draft about MIG welding) and retired to isolation in my bedroom, library, and private office. My FNP set me up with some Paxlovid, and I rode it out. While the symptoms were the same as in the previous two cases, they were not as severe. Something was a little different, though, because, at times, it felt like those sparing matches I had as a kid where I was squaring off against multiple attackers. 

I tested negative for COVID on Friday and ended Paxlovid on Sunday. The isolation sucked, and there were times around the rest of the house when I thought my mask was going to choke me, but the first two ``presumed'' times were far worse. Yes, I was still finding it hard to get out of bed for longer than an hour or two without going back to sleep, and thinking deep thoughts (what I do for a living) was debilitating. Still, though, this round was, at best, a ``mild to moderate'' case as far as COVID goes.

I can't help but think I tested positive this time because I had (clearly...) a higher viral load due to not getting the annual vaccine. (Note that I've had four others.) While the girls got vaccinated in the fall, it has been challenging for my wife and I to find a time when we weren't sick with something or showing some sort of symptoms. We tried to be nice little citizens and follow the guidance about vaccinating while sick. We tried to wait until we were perfectly healthy to get the vaccine, and we did not get vaccinated. 

That's some horse shit. I have a new resolution: I'm going to get the latest COVID vaccine on the 22nd and next fall's vaccine as soon as it comes out, even if they have to poke it through my hazmat suit.

What was that extra thing, though - this time when I was sick? I said it was like two extra attackers, right? Oh, yeah: I tested positive for strep throat and a sinus infection this morning, so my good buddy Augmentin and I have the next ten days together.

\section{A serious thought or two}

While the weak-sauce diatribe above may be hardly surprising... or interesting... what I really wanted to get to was how impressed I am with several important developments from the pandemic.
\begin{itemize}
\item mRNA vaccines - Ask me sometime how I know about this technology, but these vaccines are just some of the coolest tech on the planet, and - yes, thanks to COVID - some of the best-tested tech as well. I'm stoked about the new vaccines we can develop with this delivery mechanism!
\item Next-generation antivirals - Paxlovid and other COVID-age antivirals are amazing feats of bioengineering all on their own. Did you know that Paxlovid is actually two antivirals, with one preventing the metabolism of the other so that it hangs around longer (and kills more COVID)?
\item Monoclonal antibodies - I have not used monoclonal antibodies personally, but this is another amazing technology that has saved countless lives and can be adapted to other illnesses.
\item Electronic medical infrastructure - I may be wrong about this, but I feel like the pandemic completely kicked into high gear the development of new electronic medical infrastructure, including telehealth and scientific data tools. This isn't my field, but I'll say the way those who continue to work on COVID move suggests much progress has been made in data sharing, collaboration tech, and policies.
\item At-home test kits - The medical world has to figure out what to do about at-home testing, but clearly, there have been significant advances, and we are not too long from the FDA finally approving at-home tests for strep, RSV, flu, etc. Now we take COVID tests for granted but think back to before we had them. Aren't they great? And how nice it will be to call our doctor and say, ``I just tested positive for flu. Can we do something?''
\end{itemize}

I want to leave a final thought of thanks and gratitude for all the medical folks out there who have been fighting this pandemic on our behalf for four years, to my colleagues at work for their generosity in my absence, and to my lovely family who supported me while I isolated at home, most especially Paige for all her extra heavy lifting.

If you'll excuse me, now I must take my Augmentin. 

\end{document}
