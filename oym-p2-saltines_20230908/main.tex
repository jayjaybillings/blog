% Copyright 2023 Jay Jay Billings. Some rights reserved.
\documentclass{article}
\usepackage{graphicx} % Required for inserting images

\title{oym-p2-saltines}
\author{Jay Billings}
\date{September 2023}

\begin{document}

\maketitle

\textit{In my last post on my father's battle with dementia, I shared how he kept it secret for nearly twenty years. In this post, I'll discuss how his symptoms progressed.}

\section*{Crackers}

I wasn't sure if the scene in front of me was a sign that Dad had taken up a form of ``single septuagenarian grocery art,'' but the thought crossed my mind. It was pretty in a way - colorful, possessing no specific pattern in its assembly, and imposing. Arranged before me from floor to ceiling and from the corner of the hall clear into the kitchen was a 7-foot tall by 10-foot long wall of empty and neatly arranged saltine boxes blocking his back door.

It was 2008 or so. I was visiting Dad after a long break. We had argued the previous Christmas because I wanted to keep my trip short. When I eventually made it home to see the collage of Premium, Zesta, and Food Club cracker boxes, I wasn't surprised by the stack. He had already started the project the last time I visited because his back door had a broken lock. He put a few boxes in front of the door in case someone broke into his trailer. It wasn't the worst idea if you needed something simple until your lock was fixed: an intruder pops open the back door, knocks the boxes out of the way as they do, can't put them back like they originally were, and thus leaves you some evidence that someone came through the broken door. Of course, I offered to fix the lock, but Dad assured me that he had a guy who was going to fix it soon. So, I was aware of the boxes but surprised by the enormity of the installation.

Dad lived in that trailer for eight years. He never fixed the lock. Every year, the people who were going to rob him increased in number and ferocity, at least as he told it, but no one ever broke in or even opened the door.

When he moved in 2016, it became clear that Dad was also finally slowing down professionally. He still worked at my sister's car lot but winding down to only a few hours of work a day, and then he added napping at my brother's shop to his routine. He stopped traveling to auctions or really anywhere more than an hour away. That isn't entirely surprising for an \textit{87-year-old with a 65-year career}. Everyone was happy to see him take it easy a little. Still, though, we found some of it rather strange, like all the napping. Dad hated sleeping for most of his life. He was also eating far too much sugar, even for a lifelong lover of sweets who had to have his teeth removed at 25. We heard that he was telling stories about his experience during World War II in the Italian theater, but we knew he had never served in Italy - his brother did. And through it all, he developed an even deeper distrust of banks and authority. My brother and I were informed unexpectedly and brusquely in 2018 that we could pay his bills because he was ``Done with that shit.''

By 2019, my wife and I knew there weren't too many years left. It had been another one of those long stretches where I hadn't seen Dad very often. Even though we spoke every Tuesday morning by phone, I really wanted to see him. Dad agreed to meet us for his birthday at a lovely little apple orchard - William's Orchard just outside Rural Retreat VA - to pick apples and pumpkins with my family. He was more or less okay at this visit, albeit showing signs of being 90. It was clear he had some minor cognitive impairment: He was repeating himself a lot, he couldn't hear much at all, and he had some hygiene problems. He shared a laundry list of reasons why he didn't stray too far from home anymore and expressed his concern about how difficult it was to get to the orchard, which was only a few miles past the end of his usual range.

Something happened after that, which I mark as ``the beginning of major hostilities'' in Dad's war with cognitive decline and impairment. He had moved again to a rather remote house on the back of my brother's farm, and about six weeks after our visit to the orchard, he started acting strangely. I remember my brother calling me and saying, much to the embodiment of my fears and dread, ``Bro, you gotta come check him out. Something ain't right.'' Dad was behaving dangerously and being generally agitated, grouchy, irate, and even violent. He was always a tough man who would throw down as such when required, but now he was out of sorts every day. Paradoxically, he seemed physically fine, and also ill. He threw his phone at my brother and said he wasn't going to use it anymore, which was normal, but then he really did stop using it, which was quite abnormal. He was constantly talking really, really dirty too. He had even fired his pistol at a family member.

I hopped in my truck the next weekend I had free and paid Dad a visit. I found him in a ``We Buy Gold'' store selling collectibles that he had saved for his grandchildren. The storekeeper had been buying these for weeks off of Dad. The guy was none too happy when I reminded Dad, collected the watches, and moved us out the door.

Dad seemed completely different since I saw him at the orchard in several important ways, like selling collectibles. He believed that collectibles were actually ``things to be collected.'' He also wasn't telling stories anymore, which was devastating because he was always telling the most audacious, hilarious, thrilling, and more or less true stories from his life. These stories were like having 90 years of episodes of Dave Chappelle's ``When Keeping It Real Goes Wrong.'' He was staying very close to home too.

It turns out, that since our visit to the orchard, he had experienced two head injuries. First, he flipped his lawn mower on the hill in his yard, and a family member found him lying in the grass. (It wasn't clear if this was before or after the orchard, to be honest, but he hadn't mentioned it previously.) Second, he was working on his TV and fell into a corner, hitting his head on the wall on the way down. He had a lump on the back of his head that stuck out like an egg. He looked scared and uncertain - a new look for him - as he tried to explain what he had experienced. 

He had entered a vicious cycle of depression and loneliness where his everyday activities caused him to become distressed. For example, his loneliness made him hang out at places where he would argue with people, which just made him more depressed. The depression made him keep his distance, which made him lonely. He hadn't slept well in months because a train went by his house every night and woke him, leaving him feeling terrible and irritable every day. He was hearing voices in his house all night long, and he insisted that my sister and her kids ``were always in and out at all hours of the night partying upstairs...,'' even though they actually lived over an hour away and hadn't seen him in about a year. He described other people whom he believed to be in the house, including a beautiful young mother who was hiding from an abusive husband with her daughter and sitting in a chair next to me. (The chair was empty, but I knew who he was thinking about).  We had lots of discussions with and about people I couldn't see, various inanimate objects, and pictures. I convinced him to try out some hearing aids I brought so we could talk more, and, in turn, he tried to convince me that there were anti-Trump voices speaking whenever he flushed his toilet. To be honest, as a lifelong Democrat, I think he liked this group of auditory hallucinations the best. We had a good laugh about the toilet and went for lunch.

For the rest of the weekend and the weekends after that, I spent time with him. We had some hard discussions that fathers need to have with their children, and we laughed a lot. We buried the hatchet on a couple of issues, and he seemed to be less angry, less delusional, and to be hallucinating less. We managed to get him to a surprise 90th half birthday party in March 2020 where he got to eat ice cream with my daughter and see most of his family at once. These tough conversations and reunions were just in time, as will shortly become apparent.

I will believe until my dying days that this intervention and these visits helped him. He seemed to be  slowly getting back to how he was at the orchard. He even seemed cheerful, less depressed, and less lonely some days. He seemed happier, he had let go of some things, and the progression of his symptoms seemed to slow. He was resting more getting a break from some of the stressors. It wasn't perfect, but it was better.

Unfortunately, my visits ended too soon as the SARS-COV-2 virus spread across the world. We all went into lockdown. I called Dad's house on Tuesday, as usual, and he asked what the hell was going on because he had just returned from the local Walmart where he helped a woman fill her car with toilet paper. He insisted that I not visit him anymore - that ``I protect my girls'' - because he knew my wife was pregnant with our second child and didn't want our oldest daughter to get sick either.

In July 2020, a few months after that phone call, he had back-to-back strokes after walking two miles on the hottest day of the year to where he used to live to get something out of the garage. I spoke to Dad the next morning and he described the strokes in detail. He was really lucid and clear during that discussion, but after that, his symptoms progressed faster than the freight train keeping him awake at night. He recovered enough to start wandering again after about six months, at which time in March 2021 he was picked up, taken to the emergency room by the county police, and soon committed to the state hospital under an Emergency Custody Order. It was almost exactly a year since I last saw him. 

When he was admitted to the hospital it took five people to strap him to his bed. He told everyone he encountered that he was going to fire them because he was the governor of the world, he was in charge, and he owned everything. His mind was a conflagration fueled by his outage at being hospitalized. The doctors had no idea what was going on because he didn't ``act normal for this type of patient.''

I appeared in court with my brother in August 2021 to petition to be guardians and conservators for Dad. A few hours later I visited him at the care facility where he was staying. He didn't recognize me, but he thought we must be kinfolk. He was waiting on his son who had just called and was supposed to arrive at any moment. He showed me a picture of my family so I would know who he was waiting on. When I held it up beside my face he finally saw that I was right in front of him, although he would hardly admit to not recognizing me. 

It was my fault: I sure had ``got fat'' since he last saw me, and he'd recognize me better if I was skinnier! \#fml

\section*{Symptom Synopsis}

I find the narrative above to be as hard to read as it was to write. This is sad, confusing, and disturbing stuff, even if you have never gone through it and even if it was as tamed down as what I wrote above. I wanted to share this level of detail to illustrate how Dad's condition evolved over time, and particularly how it accelerated over the last two years of his life. This is important because reading a case study is more effective than reading only a list of symptoms from some really good website like MayoClinic. For example, I didn't really understand what a delusion looked like, or even how it was different than a hallucination, in a real patient until doctors explained it to me with examples of Dad's behaviors. This is despite the fact that I had read about delusions on multiple websites.


I also think it is important to see a case study to understand the lists of symptoms provided in other resources.  The table below correlates my father's behaviors at different times with standard descriptions of dementia symptoms. 

table

\section*{Next Time}

Thanks for reading this second part of a series about my father’s struggles with cognitive decline and what we did to help him. I hope you’ll enjoy the next part, in which I'll discuss ways families can get help and support their loved ones.

%First, no two patients with dementia are the same. Each patient takes a different path and develops a unique expression of symptoms. Second, whether it starts abruptly due to head trauma or is a slow burn, dementia evolves over time. Third, some degree of mild cognitive impairment might be easy to accommodate. Fourth, and this is the important one, you can help your friend or family member at any and all stages of dementia. They are just as lovable on the last day as they were on day 1. My Dad was still ``Jack'' right until the end.

%Basically: Here's how it looked behind the scenes, here's how he masked it, and here's what we wished we would have known about public support. Including that social services isn't out to get you.

%No stories, physical symptoms, depression, strong delusions, strong hallucinations, violent behavior, extremely dirty language, poor decision making, bad hygiene, trouble counting anything but money.

%oss of business acumen, ability to plan, trouble getting home. Problems with authority. Integrity issues.

%Repetitive speech, stronger delusions, personal hygiene.

\end{document}
