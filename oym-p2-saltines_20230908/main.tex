% Copyright 2023 Jay Jay Billings. Some rights reserved.
\documentclass{article}
\usepackage{graphicx} % Required for inserting images
\usepackage{hyperref}
\usepackage{tabularx}

\title{Outliving your mind, part 2: Saltines}
\author{Jay Jay Billings, Ph.D.}
\date{\today}

\begin{document}

\maketitle

\textit{In my \href{https://jayjaybillings.com/2023/07/06/outliving-your-mind-part-1-the-secret/}{last post} on my father's battle with dementia, I shared how he kept it secret for nearly twenty years. In this post, I'll discuss how his symptoms progressed.}

There are many excellent websites that offer detailed descriptions of symptoms and timelines for various conditions and diseases that cause cognitive decline and impairment. However, I often find it difficult to correlate that content to what I witnessed with Dad or other friends and family. Like in the case of a family friend who shoved his caregivers aside, jumped on a lawnmower, flipped the bird, and yelled ``You'll never catch me, mother@\%\&\#\^\@*s!!!'' as he rode into the sunset. Which set of symptoms on \href{https://www.mayoclinic.org/diseases-conditions/dementia/symptoms-causes/syc-20352013}{the Mayo Clinic dementia page} covered that behavior?

I offer the narrative below as a sort of case study of the way cognitive impairment progresses, or at least how it happened for my father, and how dementia can manifest in an individual. I found the quick evolution of Dad's condition during the last three years of his life to be especially important because it was the most challenging, confusing, and undocumented phase. Following the narrative, I'll discuss how Dad's symptoms aligned with more standard definitions. Finally, I'll describe how cognitive and behavioral changes are detected by professionals.

\section*{Saltines}

I wondered if the scene in front of me was a sign that Dad had taken up a form of ``single septuagenarian grocery art.'' It was pretty in a way - colorful, possessing no specific pattern in its assembly, and imposing. Arranged before me from floor to ceiling and from the corner of the hall clear into the kitchen was a 7-foot tall by 10-foot long wall of empty and neatly arranged saltine boxes blocking his back door.

It was 2008 or so. I was visiting Dad after a long break. We had argued the previous Christmas because I wanted to keep my trip short. When I eventually made it home to see the collage of Premium, Zesta, and Food Club cracker boxes, I wasn't surprised by the stack. He had already started the project the last time I visited because his back door had a broken lock. He put a few boxes in front of the door in case someone broke into his trailer. It wasn't the worst idea if you needed something simple until your lock was fixed: an intruder pops open the back door, knocks the boxes out of the way as they do, can't put them back like they originally were, and thus leaves you some evidence that someone came through the broken door. Of course, I offered to fix the lock, but Dad assured me that he had a guy who was going to fix it soon. So, while I was aware of the ``cracker castle,'' I was surprised by how far he had taken it.

Dad lived in that trailer for eight years. He never fixed the lock. Every year, the people who were going to rob him increased in number and ferocity, at least as he told it, but no one ever broke in or even opened the door.

When he moved in 2016, it became clear that Dad was also finally slowing down professionally. He still worked at my sister's car lot but had wound down to only a few hours of work a day. Then he added napping at my brother's shop to his routine. He stopped traveling to auctions or really anywhere more than an hour away. No one was surprised, given his age, and nearly everyone was happy to see him take it easy a little. Still, though, we found some of it rather strange, like all the napping. Dad hated sleeping for most of his life. He was also eating far too much sugar, even for a lifelong lover of sweets who had to have his teeth removed at 25. We heard that he was telling stories about his experience during World War II in the Italian theater, but we knew he had never served in Italy - his brother did. And through it all, he developed an even deeper distrust of banks and authority. My brother and I were informed unexpectedly and brusquely in 2018 that we could pay his bills because he was ``Done with that shit.''

By 2019, my wife and I knew there weren't too many years left. It had been another one of those long stretches where I hadn't seen Dad very often. Even though we spoke every Tuesday morning by phone, I really wanted to see him. Dad agreed to meet us for his birthday at a lovely little apple orchard - William's Orchard just outside Rural Retreat, VA - to pick apples and pumpkins with my family. He was more or less okay at this visit, albeit showing signs of what I described at the time as ``just being 90.'' It was clear he had some minor cognitive impairment: He was repeating himself a lot, he couldn't hear much at all, and he had some hygiene problems. He shared a laundry list of reasons why he didn't stray too far from home anymore and expressed his concern about how difficult it was to get to the orchard, which was only a few miles past the end of his usual range.

Something happened after that, which I mark as ``the beginning of major hostilities'' in Dad's war with dementia. He had moved again to a rather remote house on the back of my brother's farm, and about six weeks after our visit to the orchard, he started acting strangely. I remember my brother calling me and saying, much to the embodiment of my fears and dread, ``Bro, you gotta come check him out. Something ain't right.'' Dad was behaving dangerously and being generally agitated, grouchier than usual, irate, and even overtly violent. He was always a tough man who would throw down as such when required, but now he was out of sorts every day. Paradoxically, he seemed physically fine and also ill. He threw his phone at my brother and said he wasn't going to use it anymore, which was kinda normal, but then he really stopped using it, which was quite abnormal. He was also using extremely sexually explicit and vulgar language as well as engaging in other alienating behaviors. He had even fired his pistol at a family member.

I hopped in my truck the next weekend I had free and paid Dad a visit. I found him in a ``We Buy Gold'' store selling collectibles that he had saved for his grandchildren. The storekeeper had been buying these for weeks off of Dad. The guy was none too happy when I stopped Dad, collected the watches, and moved him out the door.

Dad seemed completely different since I saw him just three months earlier at the orchard in several important ways, like selling collectibles. Dad believed that collectibles were actually ``things to be collected,'' and he never sold them. He was staying even closer to home, too. He had altogether stopped telling stories, which was devastating because he was always telling the most audacious, hilarious, thrilling, and more or less true stories from his life. It was like he had a 90-year supply of episodes of Dave Chappelle's ``When Keeping It Real Goes Wrong.'' And they were just gone - he couldn't remember them anymore.

It turned out that since our visit to the orchard, he had experienced two head injuries. First, he flipped his lawn mower on the hill in his yard, and a family member found him lying in the grass. (To be honest, it wasn't clear if this was before or after the orchard, but he hadn't mentioned it previously.) Second, he was working on his TV and fell into a corner, hitting his head on the wall on the way down. He had a lump on the back of his head that stuck out like an egg. He looked scared and uncertain - a very new look for him - as he tried to explain what he had experienced. 

He had entered a vicious cycle of depression and loneliness where his everyday activities caused him to become distressed. His loneliness made him hang out at places where he would argue with people, which just made him more depressed. The depression made him keep his distance, which made him lonely. He hadn't slept well in months because a train went by his house every night and woke him, leaving him feeling terrible and irritable every day. He was hearing voices in his house all night long, and he insisted that my sister and her kids ``were always in and out at all hours of the night partying upstairs...,'' even though they actually lived over an hour away and hadn't seen him in about a year. He described other people whom he believed to be in the house, including a beautiful young mother who was hiding from an abusive husband with her daughter and sitting in a chair next to me. (The chair was empty, but I knew who he was thinking about.)  We had lots of discussions with and about people I couldn't see, with various inanimate objects, and with pictures. I convinced him to try out some hearing aids I brought so we could talk more, and, in turn, he tried to convince me that there were anti-Trump voices speaking whenever he flushed his toilet. To be honest, as a lifelong Democrat, I think he liked those voices! We had a good laugh about the toilet and went for lunch.

For the rest of the weekend and the weekends after that, I spent time with him. We had some hard discussions that fathers need to have with their children, and we laughed a lot. We buried the hatchet on a couple of issues, and he seemed to be less angry, less delusional, and to be hallucinating less. We managed to get him to a surprise 90th half-birthday party in March 2020, where he got to eat ice cream with my daughter for the first time and see most of his family at once. These tough conversations and reunions were just before the COVID-19 lockdown started.

I will believe until my dying days that this intervention and these visits helped him. He seemed to be slowly getting back to how he was at the orchard. He even seemed cheerful, less depressed, and less lonely some days. He seemed happier; he had let go of some things, and the progression of his symptoms seemed to slow. He was resting more, getting a break from some of the stressors. It wasn't perfect, but it at least appeared to be better.

In July 2020, he had back-to-back strokes. He had decided to walk two miles on the hottest day of the year to where he used to live to get something out of the garage. I spoke to him the next morning, and he described the strokes in detail. He was lucid and clear during that discussion, but after that, his symptoms progressed faster than the freight train keeping him awake at night. He recovered enough to start wandering again after about six months. He was picked up in March 2021, taken to the emergency room by the county police, and soon committed to the state hospital under an Emergency Custody Order. 

It took five people to strap him to his bed when he was admitted to the hospital. He told everyone he encountered that he was going to fire them because he was the governor of the world, he was in charge, they worked for him, he owned everything, and they needed to stop trying to rob him. His mind was a conflagration fueled by his outrage at being hospitalized. The doctors had no idea what was going on because he didn't ``act normal for this type of patient.'' They worked with Dad for five months to stabilize him and monitor his condition, during which time they discovered a number of neurological issues, including somewhat severe vascular atrophy in his brain, that led to a number of cognitive and psychological changes. Part of their evaluation included a ``KELS test'' that evaluated his living skills. Unfortunately, the results of that test indicated that he could not be left alone and would require 24-7 care for the rest of his life.

I appeared in court with my brother in August 2021 to petition to be cooperative guardians and conservators for Dad. A few hours later, I visited him at the care facility where he was staying. He looked good, but I was surprised to see him using a walker. He didn't recognize me, although he was pretty sure we must be kinfolk. He said he was waiting on his son, who had just called and was supposed to arrive at any moment. He then asked me for the time - 1:03 PM - to confirm my imminent arrival. He showed me a picture of my family so I would know who he was waiting on. When I held it up beside my face, he finally saw that I was right in front of him, although he would hardly admit to not recognizing me. He said it was my fault: I sure had ``got fat'' since he last saw me, and he'd recognize me better if I were skinnier! He also quickly disavowed all need for the walker, quietly telling me that they made him use it because he had a little fall or two. He insisted he could walk just fine while also casually leaning over and holding the railing on the wall.

We had finally transitioned from wondering what was going on with Dad to understanding his disease and getting him the care he needed. Daddy lived another seventeen months and continued to slowly decline over that time. The decline appeared to accelerate as he got closer to the end of his life because he lost ``big'' skills like walking, being able to feed himself, and being able to swallow solid foods. He died from complications of what is known as ``Adult Failure to Thrive.'' That being said, I feel compelled to reassure all of his family and friends who are reading this that even until the end, he never forgot that he was Jack Billings, how to count money, or stopped flirting with women.

\section*{Symptom Synopsis}

Stories or case studies like these are hard to read, but they are important because they add realism - a bit of seasoning, if you will - to more clinical descriptions. For example, I didn't really understand what a delusion looked like or how it was different than a hallucination in a real patient until doctors explained it to me with examples of Dad's behaviors. This is in spite of the fact that I had read about delusions on multiple websites and in books and heard stories about them all my life. When you read that a patient, a friend, a loved one, or whoever believed that they were ``the governor of the world,'' it is more relatable than what Wikipedia describes as ``a false fixed belief that is not amenable to change in light of conflicting evidence.''

I left the narrative above at a good place to compare to more standard descriptions because they usually don't cover what happens at the very beginning or the very end of the patient's journey. The tables below correlate my father's behaviors and symptoms with a list that I composed from other sources, including the Mayo Clinic and Wikipedia. These changes may look similar in others with neurocognitive disorder, or they may not present this way at all since every case is different. There are two tables: one for cognitive changes in how the brain functions (i.e., memory loss) and another for psychological changes in the way the patient behaves (i.e., inappropriate behavior). These sets of changes do not necessarily overlap (i.e., memory loss and personality changes don't necessarily happen at the same time), nor do they necessarily happen in order (i.e., personality changes don't have to precede depression).

\begin{table}[h!]
\centering
\begin{tabularx}{ 0.8\textwidth }{ | >{\raggedright\arraybackslash}X | >{\centering\arraybackslash}X | }
 \hline
 \multicolumn{2}{|c|}{Cognitive Changes in Dad} \\
 \hline
 Memory loss & No longer telling stories, forgetting that he saved collectibles for grandchildren, repeating himself\\
 \hline
 Problems communicating or finding words & Not understanding people, repeating himself, getting angry at forgetting a word  \\
 \hline
 Trouble with visual and spatial abilities & Loss of balance and falling, scared of driving, driving dangerously \\
 \hline
 Problems with reasoning or problem-solving & Getting confused about bills, unable to fix cars anymore, losing his ``internal US map,'' stacking saltine boxes instead of fixing a broken lock \\
 \hline
 Trouble performing complex tasks & Didn't know the number for 911, could not dial the phone, could not make meals from scratch \\
 \hline
 Trouble with planning and organizing & Had trouble getting supplies, could not clean his house, and easily lost things \\
 \hline
 Poor coordination and control of movements & Loss of balance and falling \\
 \hline
 Confusion and disorientation & Wandering, lashing out at caregivers, not recognizing family  \\
 \hline
\end{tabularx}
\caption{Cognitive changes from Mayo Clinic and Wikipedia correlated with changes observed in my father.}
\end{table}

\begin{table}[h!]
\centering
\begin{tabularx}{ 0.8\textwidth }{ | >{\raggedright\arraybackslash}X | >{\centering\arraybackslash}X | }
 \hline
 \multicolumn{2}{|c|}{Psychological Changes in Dad} \\
 \hline
 Personality changes & Becoming violent towards family, withdrawn, and losing track of personal hygiene \\
 \hline
 Depression & Yes \\
 \hline
 Anxiety & Yes  \\
 \hline
 Agitation & Yes, above his normal grouchiness and seriousness \\
 \hline
 Inappropriate behavior & Transitioned from his lifelong flirting and casual dating to overt sexually explicit language and alienating behaviors \\
 \hline
 Being suspicious, known as paranoia & Constantly carrying a gun for protection and discharging it at family, stacking saltine boxes to detect intruders \\
 \hline
 Hallucinating & Seeing people in his house or things in his food that were not there, hearing voices in the toilet \\
 \hline
 Delusional & Believed he was governor of the world and in charge of everything, believing that people were stealing from him and overly distrusting his care providers \\
 \hline
\end{tabularx}
\caption{Psychological changes from Mayo Clinic and Wikipedia correlated with changes observed in my father.}
\end{table}

From my perspective, and I think this is supported by the narrative above, Dad experienced cognitive changes first over a timeline of about twenty years. It was straightforward and, in some cases, easy for him to conceal most of the cognitive changes if he noticed them. For example, he often waved off repeating himself or his trouble doing something as getting old, being tired, or being stressed. Those were good enough reasons for anyone who was also old, tired, or stressed. This contrasts with the seemingly sudden appearance of his psychological changes, which developed in just over a year and mostly in only a couple of months. He shared his feelings of depression and anxiety with me in October 2018. By October 2019, when we went to the orchard together, he seemed to be experiencing a few more things. However, by Thanksgiving - only a month later! - friends and family started calling me because of his behavior. When I saw him again in early January 2020, he was experiencing every psychological change in Table 2. 

Cognitive changes can be measured directly. Earlier, I mentioned that the state hospital used the ``KELS test'' to determine that Dad needed 24-7 care. Formally known as the Kohlman Evaluation of Living Skills test, it has been validated as a strong indicator of the need for long-term care and a high likelihood of self-neglect. \href{https://www.ncbi.nlm.nih.gov/pmc/articles/PMC2855540/}{See for reference this study by Pickens et al. and published in the Journal of the American Academy of Nurse Practitioners.} This test measures both basic and ``instrumental'' activities of living. In their paper, Pickens et al. define basic activities as ``eating, bathing, dressing, and toileting,'' and instrumental activities as more complex tasks like ``preparing meals, performing housework, managing finances, or using the telephone.'' Performing these activities requires a certain degree of cognitive function, such as being able to remember that 911 is the number to call in an emergency and/or even being able to identify the existence of an emergency. Thus, the performance of a patient on a KELS test indicates the level of assistance they need as well as the current state of their cognitive abilities. 

Determining what is a normal behavior versus a psychological change can be extremely complicated, especially if the changes could be explained by other coincident life experiences. It is tempting to retrospectively label Dad's growing grouchiness and increasingly over-the-line ``flirting'' from before 2018 as early signs of irritability, agitation, and inappropriate behaviors caused by dementia. That may be true, but he was also experiencing lots of normal life changes for a man in his 80s at the same time. So was he getting agitated because he was psychologically changing due to neurological disease, or was he getting pissed off that he was burning through his assets instead of working to make money because his weakening knees and aging bladder didn't let him drive as much anymore? It is extremely important to avoid jumping to conclusions that might impact someone's quality of life or remove their autonomy and personal freedoms without first examining their situation. And, of course, on the other hand, there is a point where it becomes apparent that there are deeper problems - as it did for us in 2019 - when the changes in behavior are so absurd that the only rational conclusion can be that something is wrong.

What to do once it is clear that something is wrong is where we will pick up the story next time.

\section*{Next Time}

Thanks for reading this second part of a series about my father’s struggles with cognitive decline and what we did to help him. I hope you’ll come back to read the next part, in which I'll discuss ways families can get help to support their loved ones, including public and private organizations, social support services, legal support for elder care, how and when to engage law enforcement, and how to have tough conversations with family members.

I would especially like to thank my wonderful wife Paige for her support, Paige and my friend Brandon Nipper for their reviews of my draft, Michael Phelan for encouraging me to finish this article, and those who reached out after the first article with kind words of encouragement and gratitude. 

\end{document}
