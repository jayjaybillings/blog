% Copyright 2023 Jay Jay Billings. Some rights reserved.
\documentclass{article}
\usepackage{graphicx} % Required for inserting images

\title{oym-p2-saltines}
\author{Jay Billings}
\date{September 2023}

\begin{document}

\maketitle

\textit{In my last post on my father's battle with dementia, I discussed how it was a secret that he kept for nearly twenty years. In this post, I'll discuss how his symptoms progressed and how families can get help.}

I wasn't sure if the scene in front of me was a sign that Dad had taken up a form of "single septuagenarian grocery art," but the thought crossed my mind. It was pretty in a way - colorful, possessing no specific pattern in its assembly, and imposing. Arranged before me from floor to ceiling and from the corner of the hall clear into the kitchen was a 7 foot tall by 10 foot long wall of empty saltine boxes.

It was 2008 or so. I was visiting Dad after a long break. We had argued the previous Christmas because I wanted to keep my trip short. I wanted to see him and rest because I was exhausted from finishing my masters degree, and I still had to submit corrections in a week. Dad wanted me to work for free for a couple of weeks, buying, selling, hauling, and moving cars. He told me that if I wasn't going to work (for free...), I might as well not come home. It was going to snow anyway, he snapped at me before hanging up the call. He was implying that I couldn't drive in snow - a grave insult in my family! - even though he taught me. I called him back to say, "Sure, I'll stay in Knoxville. The weather looks terrible!" and we didn't see each other for a while.

When I eventually made it home to see the collage of Premium, Zesta, and Food Club cracker boxes, I wasn't surprised that he had stacked cracker boxes. I was surprised by how many cracker boxes he had and how massive the installation was. He had already started the project the last time I visited because his back door had a broken lock. He put a few boxes in front of the door in case someone broke into his trailer. It wasn't the worst idea if you needed something simple until your lock was fixed: an intruder pops open the back door, knocks the boxes out of the way as they do, can't put them back like they originally were, and thus leaves you some evidence that someone came through the broken door. Of course, I offered to fix the lock, but Dad assured me that he had a guy who was going to fix it soon.

Dad lived in that trailer for eight years, well into his eighties, and he never fixed the lock. Every year the people that were going to rob him increased in number and ferocity, at least as he told it. No one ever broke in or even opened the door.

Around the same time that he moved in 2016, it became clear that Dad was also finally slowing down professionally. He still worked at my sister's car lot, but slowly wound down to a few hours of work a day, and then he added napping at my brother's junkyard to his routine. He stopped traveling to auctions or really anywhere more than an hour away. He seemed to be retiring to a quiet life, and many of his friends would pull me aside when I visited to share how happy it made them that he was getting to take it easy.

We saw some strange things too, like all the napping. Dad hated sleeping for most of his life. He was also adding a lot of sugar to his diet. We heard that he was telling stories about his experience during World War II in the Italian theater, but we knew he had never served in Italy - his brother did. And through it all he developed an even deeper distrust of banks and authority. My brother and I were informed that we could pay his bills because he was "Done with that shit."

By 2019, my wife and I knew there weren't too many years left. It had been another one of those long stretches where I hadn't seen Dad very often, this time because of my Ph.D. thesis instead on my M.S. Even though we spoke every Tuesday morning by phone, I really wanted to see him. There's a lovely little apple orchard just outside Rural Retreat VA called Williams' Orchard and Dad agreed to meet us there on his birthday to pick apples and pumpkins with my oldest daughter, Ada. This was the first of only two times that they met each other, but after some sidelong glances from her and some smiles from her Papaw - plus a bunch of funny faces and pumpkins - they were cool.

Dad was mostly okay at this visit, albeit showing signs of turning 90. It was clear he had some minor cognitive impairment. He was repeating himself a lot, and he couldn't hear much at all. He had some hygiene problems. He shared a laundry list of reasons why he didn't stray too far from home anymore and expressed his concern about how difficult it was to get to the orchard, which was only a few miles past the end of his usual range. I was sad to see him go after our trip to the orchard. We followed quietly behind him in our Explorer as he walked to his Cadillac. I figured he couldn't hear us anyway, and I wanted to make sure he didn't fall. I shared with my siblings my opinion that he was obviously a little demented, but on the whole, he made 90 look good.

And then something happened. He had moved again to a rather remote house on the back of my brother's farm and a few weeks after our visit at the orchard he started acting strange. First, he threw his phone at my brother, Jack, and said he wasn't going to use it anymore. The throwing was normal, but he really did stop using it. Then, he just seemed ill according to Jack and others. He was fine physically but his mind wasn't right. He had even fired his pistol at a family member. Jack was kind enough to get Dad on the phone with me at Jack's office and Dad seemed irate in a way that was uncharacteristic even for a man known for his ability to be supremely grouchy. Jack and I talked it over, and I heard words that I had dreaded for many years: "Bro, you gotta come check him out. Something ain't right."

I packed my truck with what I thought Dad would need based on my limited and naive understanding of dementia. The loot included some \$50 hearing amplifiers from Amazon.com, a homemade eye chart, a collection of household supplies, an army surplus red landline telephone for his house, a baby monitor to spy on him, and fresh beans and cornbread from my wife. I also brought with me a fair amount of resolve and determination to help him.

He seemed completely different since I saw him at the orchard in several important ways. He was still Dad, of course, but he wasn't telling stories anymore. This was devastating because Dad was always telling the most audacious, hilarious, thrilling, and more or less true stories from 90 years of keeping it real. I was even more surprised when I found Dad in a "We Buy Gold" store selling collectibles that he meant for his grandchildren to have, which he couldn't remember. I pissed off the shop owner by offering to buy the gear from Dad before he could sell it.

We spent the next two days together where he revealed to me since we last saw each other that he had experienced two head injuries. First, he flipped his lawn mower on the hill in his yard, and a family member found him lying in the grass. Second, he was working on his TV and fell into a corner, hitting his head on the wall on the way down. He had a lump on the back of his head that stuck out like an egg. He looked scared and uncertain, which was a new for him, as he tried to explain what he had experienced. 

He was depressed and lonely, which made him hang out at my brother's shop, but that only made them argue more, which made Dad more depressed and lonely. He hadn't slept well in months because a train went by his house every night and woke him. Also, he was hearing voices in the house all night long, and he insisted that my sister and her kids "were always in and out at all hours of the night partying upstairs...," except they actually lived over an hour away. There were others in the house, he said, including a beautiful young mother who was hiding from an abusive husband with her daughter and sitting in a chair next to me (it was empty, but I knew who he was thinking about).  We had lots of discussions with and about people I couldn't see, various inanimate objects, and pictures. I convinced him to try out the hearing aids I brought so we could talk more, and he tried to convince me that there were anti-Trump voices speaking whenever he flushed his toilet. To be honest, as a lifelong Democrat, I think he liked this group of auditory hallucinations the best.  We had a good laugh about the toilet and went for lunch.

For the rest of the weekend and the weekends after that, I spent time with him. We had some hard discussions that fathers need to have with their children, and we laughed a lot. We buried the hatchet on a couple of issues, and he seemed to be less angry, less delusional, and to be hallucinating less. We managed to get him to a surprise 90th half birthday party in March 2020 where he got to eat ice cream with Ada and see most of his family at once.

I will believe until my dying days that this intervention and these visits helped him. He slowly seemed to be getting back to how he was at the orchard. That wasn't perfect, but it was better. He seemed happier, he let go of some things, and the progression of his symptoms seemed to slow. 

Unfortunately, my visits ended too soon. The SARS-COV-2 virus spread across the world and we all went into lockdown. I called Dad's house on Tuesday, as usual, and he asked what the hell was going on. He has just returned from the local Wal-Mart where he helped a woman fill her car with toilet paper. I explained the situation and told him I wasn't sure when I could travel again. He told me not to worry about it because I had one duty as a husband and a father during a time like this: I had to protect my growing family.

In July 2020, a few months after that phone, he had back-to-back strokes. He decided to walk about two miles on the hottest day of the year to where he used to live to get something out of the garage. My brother owned it, so my Dad just left all his stuff there when he moved. A neighbor found Dad collapsed in the garage and took him back to my brother's. I spoke to Dad the next morning and he described the strokes in detail. He was really lucid and clear during that discussion, but after that his symptoms progressed faster than the freight train that kept him awake at night. He recovered enough to start wandering, at which point he was picked up and committed to the state hospital by local law enforcement. That was almost exactly a year since I last saw him. 

When he was admitted it took five people to strap him to his bed. He told everyone he encountered that he was going to fire them because he was the governor of the world, he was in charge, and he owned everything. His mind was a conflagration, and the doctors had no idea what was going on.

I appeared in court with my brother in August 2021 to petition to be guardians and conservators for Dad. A few hours later I visited him at the care facility where he was staying. He didn't recognize me, but he thought we must be kinfolk. He was waiting on his son who had just called and was supposed to visit him any minute. He showed me a picture of my family so I would know who he was waiting on. When I held it up beside my face he finally saw that I was right in front of him. 

He wouldn't admit to not recognizing me. It was my fault: I sure had "got fat" since he last saw me, and he'd recognize me better if I was skinnier! \#fml

\section*{Symptom Synopsis}

I imagine the story above was as hard to read as it was to write. I wanted to share this level of detail because it is hard to find. It illustrates

\section*{Resources}

\section*{Next Time}

%Basically: Here's how it looked behind the scenes, here's how he masked it, and here's what we wished we would have known about public support. Including that social services isn't out to get you.

%No stories, physical symptoms, depression, strong delusions, strong hallucinations, violent behavior, extremely dirty language, poor decision making, bad hygiene, trouble counting anything but money.

%oss of business acumen, ability to plan, trouble getting home. Problems with authority. Integrity issues.

%Repetitive speech, stronger delusions, personal hygiene.

\end{document}
